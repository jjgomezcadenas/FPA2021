\subsection{Objectives of the \sLSC\ subproject}
\label{sec.obj.lsc}

LSC takes three major roles in this project: {\bf i)},  leads the design and construction of the neutron shielding and muon veto 
for NEXT-100 and the operation of NEXT-100 muon veto (in collaboration with DIPC and IFIC); {\bf ii)}, coordinates the design and logistics of the 
screening campaign (in collaboration with all teams) for NEXT-HD, and {\bf iii)}, leads the design
and construction of the water tank and the inner copper shielding for NEXT-HD. In addition, LSC will 
purchase the xenon (both enriched and depleted  in \XE) needed for \NHD. 

\subsubsection*{Muon veto and neutron shielding for NEXT-100}

NEXT-100 sensitivity will be ultimately limited by the flux of muons reaching the underground lab and the environmental neutron flux 
underground. Both are well measured in the lab and require monitoring and active or passive reduction. The muon veto design has been described in the previous sections. LSC will be in charge of purchasing the materials and in-place assembly. In addition, a polyethylene castle will be installed around the 
detector to reduce the impact of environmental neutrons.

\subsubsection*{Coordination of screening program for NEXT-HD}

One of the lessons learned during the construction of NEXT-100 is that the logistics of the screening campaign of all components must be carefully designed, in order to select providers and batches of low activity materials. An example to illustrate this point is the need to acquire forty five tons of ultra radiopure copper for the inner copper shield of \NHD. This implies locating one or more providers able to supply large batches of radiopure material, reserve and test samples of such batches, perform ultra-sensitive activity measurements, select suitable batches and the follow the machining process to ensure that no radioactivity is added down the line. Just this effort requires a full qualified FTE, but the same activities need to be carried out for the steel (or titanium) of the PV, the PTFE of the DSPs, the fibres of the BFD, the copper, resistors and HDPE of the field cage, etc. While all the above was carried out for the construction of \Next\ the larger scale of \NHD\ imposes a much stricter logistics. Furthermore, LSC owns the germanium detectors and ICPM apparatus that will be devoted to the screening measurements.

\subsubsection*{Infrastructures for NEXT-HD}

The competition in double beta experiments and the efficiency of the activities of the NEXT collaboration require that the NEXT-HD infrastructures are fully functional no later than 2025, which in turn implies that design work must start already in 2022.  Notice that 
the design of such infrastructures, in particular the interfaces to the NEXT-HD detector, requires careful planning, which in turn implies a time allocation that must be added to that needed for tendering, procurement and installations. It follows that the work towards the 
design and construction of the water tank and the ICS for NEXT-HD must start already at the beginning of this project.  
%In parallel, the LSC will purchase 
%enriched and depleted xenon on Xe-136, which is a difficult operation for a public lab and will require a dedicated person in charge to succeed in the 
%procurement of xenon on time.\\

\subsubsection*{Personnel in the Working Plan}

The working plan for INFR involves 5.5 FTEs (C. Pe\~na Garay, L. Cid, S. Borjabad, A. Bayo, S. Fern\'andez, B. Hern\'andez,
R. Hern\'andez, F. Gimeno, V. Gim\'enez). With the exception of CPG and L. Cid  all members of the team are accounted 0.5 FTE 
each. They are not included in other AEI projects but serve to the experiments and activities hosted at the LSC. Other LSC members 
will support NEXT activities as an LSC service (gamma ray screening, ICPMS measurement, neutron background control, maintenance 
and safety) but are not included in the FTEs. The responsibilities on the project of the existing personnel are:

\begin{itemize}[noitemsep,topsep=0pt,parsep=0pt,partopsep=0pt]
\item C. Peña Garay is the Director of the Laboratorio Subterr\'aneo de Canfranc. He is in charge of the overall coordination of the INFR sub-project.
\item L .Cid is an expert in mass spectrometry and will be leading the R\&D of the screening campaign, where is mandatory to improve current 
technologies in the lab, from the current ppt sensitivity to the the NEXT-HD requirement of O(10) ppq sensitivity.
\item  S. Borjabad is an expert in highly pure copper and will be supporting the research of the copper procurement campaign, one of the primary 
radioactive components which needs to be improved from the NEXT-100 copper purity to the the NEXT-HD requirement.
\item A. Bayo is the LSC expert in electronics and will be involved in the design and the installation of the electronics of the muon veto 
of NEXT-100 and the water tank of NEXT-HD, and the slow-controls associated to these installations.
\item S. Fern\'andez is the LSC expert in IT and will be involved in the design and installation of the computing of the muon veto of NEXT-100 
and the water tank of NEXT-HD.
\item A. N\'u\~nez is a senior engineer. She is posted at the LSC, and her roles is acting as GLIMOS  for \Next\ and \NHD. 
\item B. Hern\'andez  is a technician, and will assist in the design, construction and physics validation the muon veto and the water tank.
\item R. Hern\'andez  is a technician, and will assist in the research for new methods of purification and cleaning of surfaces of components for NEXT-HD.  
\item F. Gimeno is a technician, and will assist in the construction (mechanical) of the polyethylene castle, the muon veto of NEXT-100 and 
the water tank of NEXT-HD
\item V. Gim\'enez is a technician, and will assist in the construction (electrical) of the muon veto of NEXT-100 and the water tank of NEXT-HD.
\end{itemize}

The LSC will purchase the materials needed for the \Next\ muon veto as well as the infrastructures for \NHD, including the xenon. For this project, the LSC requests one physicist (or engineer) position to lead the coordination of the screening campaign (screening campaign coordinator, SCC), and one mechanical engineer to work in the design and construction of the infrastructures, specifically the water tank and the ICS (infrastructures engineer, IE). Notice that SCC and IC must collaborate closely for the acquisition of the ultra-pure copper. At the same time, the IE must work closely with the integration managers from NEXT. 

This two positions are essential for the development of the project, since the LSC cannot provide them from its own personnel, given the high level of specialisation needed for the tasks, and the limited personnel available in the lab. At the same time, the cost of the positions, requested to the AEI is a small fraction (ess than 10 \%) of the resources invested by the LSC in the project. 
  
