\subsection{Budget request}

%This coordinated project requires funding for the continuation of a 10 year-long effort in the NEXT-program.
%
%The infrastructures already existing for the experiment as well as the \NEW\ detector have been co-funded by the CONSOLIDER-INGENIO project CUP (2009-2014), which provided 5 million \euro, and an AdG/ERC project granted to G\'omez-Cadenas (2014-2019), which provide 2.8 million \euro. The LSC has contributed to many of the infrastructures and also owns the enriched xenon that will be used by \Next\, whose estimated cost today exceeds 2 million \euro\ (only three groups in the world own large amounts of enriched xenon, KamLAND-Zen experiment, EXO experiment and the LSC). 
%  
% The infrastructures already available at the LSC include: the working
% platform and seismic table (LSC infrastructure); the lead castle (LSC
% infrastructure); a radon abatement system (LSC infrastructure) that
% injects radon-free air into the volume surrounding \Next, among other experiments; a clean tent (LSC infrastructure) surrounding all the experiment; a sophisticated gas system (AdG/ERC as well as contributions from Portugal and USA); a large emergency evacuation tank (AdG/ERC); electrical, vacuum and pressure equipment such as UPSs, pumps and gaskets (AdG/ERC); the \NEW\ detector (AdG/ERC funds); the \Next\ pressure vessel (purchased using CUP funds); a large fraction of the detector PMTs were also purchased using CUP funds.
 
 This coordinated project requires funds for:
 \begin{itemize}[noitemsep,topsep=0pt,parsep=0pt,partopsep=0pt]
\item Scientific exploitation of the \Next\ detector (including costs of operation at the LSC). 
%\item Computing for \Next\ at the LSC.
\item Maintenance of the \Next\ infrastructures (mainly the gas system). 
\item Radioactive sources for calibration. 
\item Construction of the \HDEMO\ prototype. 
%\item Preparation of the infrastructures for \NHD.
\item Essential personnel. 
\item Travel other than the LSC (NEXT workshops and attendance to international conferences).
\item Others (small equipment, maintenance of local laboratories, etc.). 
\end{itemize}

\subsubsection{Costs of operations at the LSC}
\label{sec:operationcost}
The costs of \Next\ operations at the LSC, will be distributed among the four groups participating in the project. The costs are estimated after five years of operation of \New\ and are, therefore, very accurate. They include rental of an apartment plus appliances (1,000 \euro/month, which is much cheaper than lodging personnel in hotels), travel and allowances for shifters and engineers.  The costs of the apartment are assigned to DIPC. The cost of travel and allowances is distributed among the four groups taking into account the personnel traveling to LSC in each group. The cost of travel is computed assuming that the group in charge of the shifts (rotates every week) rents a car ---rental cars are the only practical way to arrive to Canfranc, and to access the laboratory---. The cost of the cheapest rental car is estimated as 400 \euro/week. Concerning maintenance, we keep it to a minimum (50 \euro\ per day and person), which covers meals while at LSC. Therefore, we estimate 750 \euro\ per week and shifter. Two shifters are needed for regular operations, 40 weeks per year. The total number of person-shifts is 80, which adds up to 60,000 \euro/year (180,000 \euro\ for 3 years).

The shifts are run by the whole collaboration, but the burden of shifts at LSC is taken by the Spanish and Portuguese groups, while the remote shifts are mostly run by US and Israel groups. After the experience running \NEW\, we have found that the four groups participating in this project run 40\% of the shifts at the LSC, the rest of the Spanish and Portuguese groups run 40\% and the US and Israel groups run the remaining 20\% and most of the remote shifts. This results in a total of 
72,000 \euro\ for shifts. Given the relative size of the groups, we request 20,000 \euro\ for DIPC and IFIC and 16,000 \euro\ for USC and UPV.

\subsubsection{Costs of maintenance of \Next\ infrastructures}

The costs of maintenance, specified in table \ref{tab.main}, are dominated by the gas system and are accurately estimated after the long operation of \NEW. Maintenance is a must for the proper performance and safety of the experiment. Since IFIC is in charge of the maintenance of the gas system, this cost is assigned to the \sIFIC\ subproject.

\begin{table}[h!]
\begin{center}
\begin{tabular}{|l|r|r|r|}
\hline
Part & 	Cost (\euro/year)	& Years	& Subtotal (\euro) \\
\hline
Pump maintenance	& 1000	& 3	& 3000\\
Compressor              & 4,000 &	3	& 12,000 \\ 
Repair of equipment	& 5,000 &	3	&15,000 \\
Cleaning products and operations	& 2,000 &	3 & 6,000 \\
Gas supply (Ar, He, N$_2$)  & 5,500 & 	3 &	16,500 \\
Electronics spares & 4,000   & 3 & 12,000\\
Miscellanea (gaskets, valves, spares)	& 6,000 &	3 & 18,000 \\
Tooling	& 8,000 & 3 & 24,000 \\
\hline
{\bf Total} & & & {\bf 106,500}\\
\hline
\end{tabular}  
\caption{Maintenance of \Next\ (\sIFIC).}
\label{tab.main}
\end{center}
\end{table} 


\subsubsection{Costs of radioactive sources for calibration}

Another fixed cost is that of radioactive sources. Table \ref{tab.calib} details the cost of the conventional sources (\Kr{83} and \Th{228}). Since USC is in charge of calibration, this cost is assigned to the \sUSC\ subproject.

\begin{table}[h!]
\begin{center}
\begin{tabular}{|l|r|r|r|}
\hline
Part & Cost (\euro/unit) & Units & Subtotal (\euro) \\
\hline  
Kr-83 &	3,000 &	3 & 9,000 \\
Th-228 &	12,000 & 1 & 12,000 \\
\hline
{\bf Total }   &  & &  {\bf 21,000} \\
\hline
\end{tabular}  
\caption{Radioactive sources for calibration (\sUSC).}
\label{tab.calib}
\end{center}
\end{table}

\subsection{Other travel, small equipment and lab maintenance}
\label{sec:othercosts}

NEXT holds two Collaboration Meetings (CMs) per year (December and May) at the LSC. The dates are adjusted to the meeting of the LSC Scientific Committee, which takes 2 days, and the meeting takes 2 days right before the LSC-SC meeting. We request funding for attendance of the members of the project to the CMs, at a very adjusted budget of 400 \euro/week and person.

The NEXT Collaboration is regularly invited to present results in international conferences. We request funds for two conferences during the three years for senior members and post-docs, and one conference plus one school during the three years for students.

Finally, each group requests a modest amount of funding for small equipment and maintenance of laboratories at home institutions (10,000 \euro\ per group and year). 

%\subsubsection{Costs of computing at the LSC}

\subsubsection{DIPC}

DIPC brings considerable resources to the project. The total number of FTEs working in the project is 9. DIPC has provided a large laboratory for \HDEMO\ and the funds to equip it with a state of the art gas system (already purchased) which includes: the gas loop (50 k\euro), compressor (140 k€),
vacuum system (15k\euro), leak detector (12 k\euro), High Voltage sources (20 k\euro), and lab equipment, including scope, picoamperimeter, tools, etc 
(about 30 k\euro). In total, DIPC has invested more than 300 k\euro in the \HDEMO\ lab. 
 
DIPC requests:

\begin{itemize}[noitemsep,topsep=0pt,parsep=0pt,partopsep=0pt]
\item {\bf Costs of personnel}: PD1 (run coordinator) and PD2 (\HDEMO):  50,000 \euro\ per year and person, for a total of 300,000 \euro\ \item {\bf Operation of \Next\ at the LSC}. Cost of the apartment at Canfranc: 12,000 \euro\ per year for a total of 36,000 \euro\ and cost of shifts 20,000 \euro\ (see sec.\ref{sec:operationcost}) 
%\item Operation of \Next\ at the LSC. Shifts: 20,000 \euro.
\item {\bf Cost of the \HDEMO\ pressure vessel}: 100,000 \euro.
\item {\bf Cost of the \HDEMO\ TPC} (items are detailed in the Web formulary):  35,000 \euro.
\item {\bf Consumables for the \HDEMO\ laboratory}: 10,000 \euro.
\item {\bf Xenon for \HDEMO\ prototype}: 4 kg, at a cost of  12,000 \euro\ per kg (48,000 \euro\ total). 
\item {\bf Travel} other than the LSC (NEXT workshops and attendance to international conferences): 48,000 \euro\ for the whole team. 
\item {\bf Small equipment}: 10,000 \euro.
\end{itemize}

The total requested direct cost is 607,000 \euro. 

\subsubsection{IFIC}

IFIC brings to the project 5 FTEs and an equipped laboratory for the R\&D of NEXT-HD, where the pBFD detector will be developed and operated. The IFIC laboratory has hosted the previous NEXT prototypes and offers the infrastructures for the proposed R\&D line (clean room, gas system, vacuum vessels, dark boxes, vacuum monochromator, light sources and oscilloscopes, among others), with an approximate investment in excess of 350,000 \euro. 

The cost estimate for the \sIFIC subproject is listed below:
\begin{itemize}[noitemsep,topsep=0pt,parsep=0pt,partopsep=0pt]
    \item {\bf Personnel}: 47,000 \euro\ per year for senior electronic engineer E1 (\Next\ maintenance, muon veto FEE and DAQ and NEXT-HD R\&D), 40,000 \euro\ for postdoc PD3 (scientific exploitation of \Next\ including muon veto), for a total of 261,000 \euro  
    \item {\bf NEXT-100 maintenance}: 106,500 \euro\ (see Tab.\ref{tab.main})
    \item {\bf NEXT-100 operation (shifts)}: 20,000 \euro\ (see Sec.\ref{sec:operationcost})
    \item {\bf Consumables for the IFIC laboratory}: 10,000 \euro\ (see Sec.\ref{sec:othercosts})
    \item {\bf BFD R\&D and pBFD}: 100,0000 \euro\ in total (items are detailed in the Web formulary). 
%     divided in
%        \begin{itemize}
%            \item fibers sensors and electronics: 7,5000 \euro
%            \item optomechanic elements: 7,500 \euro
%            \item vessel for pBFD mock-up: 10,000 \euro
%            \item fibers for pBFD panels: 10,000 \euro
%            \item sensors and readout for pBFD: 10,000 \euro
%            \item mechanics and coating of pBFD: 5,000 \euro
%            \item gases (Ar, Xe, N2): 10,000 \euro
%            \item computing (hard disks, processors): 10,000 \euro
%            \item radioactive sources: 10,000 \euro
%            \item development of cold electronics for SiPMs: 13,500 \euro
%            \item general electronics: 20,000 \euro
%        \end{itemize}
    \item {\bf Attendance to Collaboration Meetings}: 19,200 \euro\ (see Sec.\ref{sec:othercosts})
    \item {\bf Conferences and schools}: 20,000 \euro\ (see Sec.\ref{sec:othercosts})
\end{itemize}

The total direct cost is 550,200 \euro.

\subsubsection{UPV}
UPV brings to the project CAD (Computer Aided Design) software licenses through Europractice foundation as long as foundry access for MPW (MultiProject Wafer) prototypes and short production series. 

The cost estimate for the ASIC development draws on estimated device areas per design iteration for a typical 180nm node process (XFAB). This includes a possible rework due to high failure risk when dealing with silicon designs (selected process offers at least two foundry windows per year). The number of dies assumes a 80\% yield. 

The cost estimate for the \sUPV subproject is listed below:
\begin{itemize}[noitemsep,topsep=0pt,parsep=0pt,partopsep=0pt]
    \item {\bf Personnel}: 47,000 \euro\ per year for a senior electronics engineer for DAQ, test boards and slow controls, and 37,000 \euro\ per year for a 
junior microelectronic engineer for silicon design. The total amounts to a total of 252,000 \euro.
\item {\bf ASIC development}: The total cost (items are detailed in the Web formulary) is 301,500 \euro.
\item {\bf DAQ development}: The total cost (items are detailed in the Web formulary) is 67,000 \euro. 
\item {\bf Travel}, including both maintenance at LSC and attendance to collaboration meetings: 29,400 \euro. 
\item {\b Conferences and schools}: 6,000 \euro.
\end{itemize}

The total direct cost is 655,900 \euro.


%\begin{itemize}[noitemsep,topsep=0pt,parsep=0pt,partopsep=0pt]
%    \item AF single channel test chip (10 mm2 - 20 units) + packaging: 17,000 €
%    \item Test Boards for test chip 1 (estimated): 3,000 €
%    \item FPGA development boards for CH prototyping (estimated): 15,000 €
%    \item AF for \HDEMO\ (20 mm2 - 80 dies) + packaging: 41,500 €
%    \item Test Boards for AF \HDEMO\ (estimated): 3,000 €
%    \item FPGA components and boards for AF and CH in \HDEMO\ 20,000 €
%    \item AF final design (40 mm2 - 100 dies) + packaging: 80,000 €
%    \item Test Boards for AF final design (estimated): 3,000 €
%    \item Extra run costs for rework: 80,000 €
%    \item FPGA components and boards for AF and CH final design 20,000 €
%    \item Europractice maintenance fees for 3 years: 9,000 €
%    \item General equipment for CAD (servers, disk storage): 10,000 €
%\end{itemize}

%The cost estimate for DAQ research and development (items are detailed in the Web formulary) is 67,000 \euro. 
%\begin{itemize}[noitemsep,topsep=0pt,parsep=0pt,partopsep=0pt]
%    \item Slow Controls - CompactRIO modules (2) + Extra hardware: 5,000 €
%    \item DAQ - Read Out (5): 20,000 €
%    \item DAQ - Chassis (1): 6,000 €
%    \item DAQ - Computing (2): 10,000 €
%    \item DAQ - Cabling: 1,000 €
%    \item DAQ - Prototyping costs: 10,000 €
%    \item Small equipment and maintenance: 15,000 €
%\end{itemize}

%The costs of personnel include a senior electronics engineer for DAQ, test boards and slow controls: (45,000 \euro\ per three year for a total of 135,000 \euro) and a 
%junior microelectronic engineer for silicon design: (45,000 \euro\ per three year for a total of 111,000 \euro).
%
%    \item Travel expenses plus diets for: 29,400 €
%    \item Conferences and Schools: 6,000
%\end{itemize}
%The total direct cost of the subproject is 641,500 €

\subsubsection{USC}
UPC brings to the project 2 FTEs and 3 Ph.D students. USC intensively use the computing resources of the ``Galician Supercomputacional Center'', CESGA. The cost for the \sUSC project is

\begin{itemize}[noitemsep,topsep=0pt,parsep=0pt,partopsep=0pt]
    \item {\bf Personnel}: a postdoc position for a technical software coordinator, 40,000 \euro\ per year, for a total of 120,000 \euro.
    \item {\bf Radioactive sources}: 21,000 \euro.
    \item {\bf Operation of NEXT-100}, shifts: 16,000 \euro.
    \item {\bf Attendance to Collaboration meetings}: 14,000 \euro.
    \item {\bf Conferences and schools}:  16,000 \euro.
    \item {\bf Computers and consumables}: 10,000 \euro.
\end{itemize}

The total direct cost is: 197,000 \euro.
\subsubsection{LSC}

LSC brings to the project 5.5 FTEs and and investment in excess of 1 M\euro (including the purchase of the \Next\ muon veto parts, and the \NHD\ infrastructures), plus the cost of the xenon. The cost for the \sLSC\ project is strictly personnel, specifically 2 senior physicist/engineer positions, for 3 years each, at a cost of 
50,000 \euro\ per year, for a total of 300,000 \euro.
 