
%\subsection{Scientific and technological impact}

This project involves the scientific exploitation of a detector unique in the world, implementing the HPXe technology with EL readout, widely considered as one of the most promising ones in the field of \bbonu\ searches. Furthermore, it proposes to prepare the infrastructures at the LSC and to carry the R\&D necessary to push the technology to the ton-scale, thus, joining the group of experiments who can make a major discovery. 

\indent

The project is bringing substantial innovation to the industry of science in Spain. The experiment uses and develops cutting-edge technology, involving national and international firms. In particular, we have built most detector parts with Spanish companies have developed R\&D  projects with a number of national and international firms.

\indent

Furthermore, the R\&D associated to Barium-tagging is opening a new inter-disciplinary field. Our approach incorporates elements of organic chemistry, atomic physics, nuclear physics, particle physics, laser-matter interactions, and photonics. It involves the technology of \HPXeEL\ chambers, ion sources, magnetic-traps, and lasers (both visible and IR). There is a clear potential to develop experimental techniques associated to our experiments that go beyond our particular needs and have wide applications. This has been acknowledged by the ERC Synergy Grant adjudicated to G\'omez-Cadenas, Coss\'io and Guenette. 

\subsubsection{Dissemination of results}

\indent
The results of the experiment will be amply advertised. NEXT is very visible in social media and we are putting together, in collaboration with LSC, science museum at ``casa de los abetos'', a refurbished building at Canfranc. In addition to science demonstrators (such as a fog chamber), the museum will have a virtual tour through the underground lab and experiments, and will display outreach posters and films. A number of art works will also be displayed, as well as a section on the history of the science at Canfranc. 

\indent

LSC and DIPC will hire jointly a person in charge of outreach and dissemination, to take care of LSC and NEXT web pages, keep the museum materials up to date and work in social media. In addition, G\'omez-Cadenas
is the science-advisor of the well known JotDown magazine, and develops an intense outreach activity which involves interviews to scientific personalities, including, recently, a Nobel Prize laureate (Kajita) and the CERN general director (Gianotti) .

\subsubsection{Technology transfer}
The NEXT project has already resulted in a promising application for society, that of a TOF-PET based in liquid xenon (PETALO). Associate IKERBASQUE professor P. Ferrario (DIPC) has been awarded an StG/ERC to develop this concept.


