
%\subsection{Scientific and technological impact}

%This project involves the scientific exploitation of a detector unique in the world, implementing the HPXe technology with EL readout, widely considered as one of the most promising ones in the field of \bbonu\ searches. Furthermore, it proposes to prepare the infrastructures at the LSC and to carry the R\&D necessary to push the technology to the ton-scale, thus, joining the group of experiments who can make a major discovery. 

%\indent

%The project is bringing substantial innovation to the industry of science in Spain. The experiment uses and develops cutting-edge technology, involving national and international firms. In particular, we have built most detector parts with Spanish companies have developed R\&D  projects with a number of national and international firms.

%\indent

%Furthermore, the R\&D associated to Barium-tagging is opening a new inter-disciplinary field. Our approach incorporates elements of organic chemistry, atomic physics, nuclear physics, particle physics, laser-matter interactions, and photonics. It involves the technology of \HPXeEL\ chambers, ion sources, magnetic-traps, and lasers (both visible and IR). There is a clear potential to develop experimental techniques associated to our experiments that go beyond our particular needs and have wide applications. This has been acknowledged by the ERC Synergy Grant adjudicated to G\'omez-Cadenas, Coss\'io and Guenette. 


\subsection{Scientific and technological impact}

Its is  widely acknowledged that the quest for the \bbonu\ is a solid strategy for contributing to the understanding about the nature of neutrino, and there is also little doubt that learning about this particle will trigger a major upgrade of the Standard Model. This will be  a major achievement of Modern Physics, and shows that the scientific impact of a successful \bbonu\ experiment will  be phenomenal. To this regard, it is worth mentioning that the XXX meeting held at Gran Sasso on September 2021, established that NEXT is the only project which currently pursuits development of technology that will eradicate the background problem, and therefore with best chances of succeeding in providing reliable and unequivocal scientific results in the mid and long term. 

In any case, showing ability to develop, build and run a \bbonu\ experiment places Spain in a world-class league with very few players. The reason underpinning such exclusivity is the rare capacity to design and implement an integrated approach, capable of solving in a coordinated manner the many scientific, technological and industrial challenges encountered while the experiment is scaled up to a relevant size. Remarkably, Spain competes with technological powers such as the US, Germany or China (which, in principle, can enable much larger resources) and can do so because NEXT has managed to consolidate a full value chain, including a critical mass of national academic and industrial partners and singular facilities such as the LSC, where the experiment can be developed up to its 1-ton version. This focused national scale effort provides leverage for pushing the project forward avoiding as many as possible critical external dependencies from foreign partners.  

This has important implications regarding the impact of the project beyond the scientific boundaries:
\begin{itemize}
\item The construction of NEXT detectors requires some national industrial partners to go beyond their own current capacities. The development of state-of-the-art technologies contributes to augment their international visibility and strengthens their showcase portfolio, helping them to diversify into markets such as those of large scientific instruments. 
\item  NEXT provides support to the aforementioned industrial partners both by sharing know-how and by collaborating with them in attracting funding. Sources of this funding can be both public competitive calls and/or private investment. Such focused investment with clear and ambitious technical targets optimises the use of funds.  
\item The successful development of NEXT will demonstrate that LSC can be a reliable long term partner, with a clear role within the global network of underground laboratories. While LSC cannot compete in size and depth with some other underground facilities, it is strategically well placed within European soil, has easy horizontal access and can adapt its resources in a flexible and reliable manner. This will rise the international profile of this national singular facility and attract new relevant scientific projects in fields such as quantum technologies, biology and particle physics. 
\item On the individual scale, it is worth stressing that NEXT unusual environment provides many young scientist, engineers and technologist with rather unique work experience. The interaction with companies eases the transfer of these professionals back and forth  between the academic and industrial environments, a process which is still a major barrier form many. 
\end{itemize}

The technological output of the project deserves a detailed look on its own. A complex instrument such as the HPXe detector implements a large number of new technologies, many of them pushed to unprecedented limits of performance. While is it possible that applications of some of these technologies remains still unforeseen, in some other cases we have already identified domains of application with important economical and societal impact: 
\begin{itemize}
\item The HPXe detector integrates technology used to detect and characterise the light events taking place within the Xe contained inside the detector. This technology can be implemented in Positron Emission Tomography instruments that will use Xe gas instead of scintillating crystals. The use of a scintillating gas, instead of an array of crystals, allows for a tomographic reconstruction with much better spatial resolution. This concept is being explored by ERC project PETALO and by a Competitive Purchase launched by Generalitat Valenciana.

\item The Ba-tagging technology, originally developed at NEXT for detection of a single Ba atom produced on the event of a \bbonu\ decay, can be used for detection of other molecules, or elements such as viruses, in extremely low concentrations. Project DETENTE, funded by the Basque Government, is currently exploring the application of this technology for the detection of COVID-19 virus. 
\end{itemize}

\subsubsection{Dissemination of results}

\indent
The results of the experiment will be amply advertised. NEXT is very visible in social media and we are putting together, in collaboration with LSC, science museum at ``casa de los abetos'', a refurbished building at Canfranc. In addition to science demonstrators (such as a fog chamber), the museum will have a virtual tour through the underground lab and experiments, and will display outreach posters and films. A number of art works will also be displayed, as well as a section on the history of the science at Canfranc. 

\indent

LSC and DIPC will hire jointly a person in charge of outreach and dissemination, to take care of LSC and NEXT web pages, keep the museum materials up to date and work in social media. In addition, G\'omez-Cadenas
is the science-advisor of the well known JotDown magazine, and develops an intense outreach activity which involves interviews to scientific personalities, including, recently, a Nobel Prize laureate (Kajita) and the CERN general director (Gianotti) .

\subsubsection{Technology transfer}
The NEXT project has already resulted in a promising application for society, that of a TOF-PET based in liquid xenon (PETALO). Associate IKERBASQUE professor P. Ferrario (DIPC) has been awarded an StG/ERC to develop this concept.


