\subsection{Rationale for coordination}
 The scientists who participate in four of the five groups presenting this proposal (IFIC, UPV, USC and DIPC) form the core of the Spanish part of the NEXT collaboration\footnote{\url{https://next.ific.uv.es/next/}}, a leading experiment searching for neutrinoless double beta decay processes using high pressure gas xenon Time Projection Chambers ({\bf HPXe-TPCs}). The groups have collaborated for more than ten years, developing the project from early prototypes to the \Next\ detector ---an HPXe TPC deploying a mass of 100 kg of xenon enriched at 90\% in the \XE\ isotope---, currently being assembled at the Canfranc Underground Laboratory (LSC).  Our groups have been partners in the CONSOLIDER-INGENIO project CUP (2009-2014), and in the coordinated projects FIS2012-37947-C04-01, FIS2014-53371-C4-1-R and RTI2018-095979-B-C1. The fifth group participating in this proposal is integrated by LSC personnel. The coordinator of this project is also the founder, co-spokesperson and executive spokesperson of NEXT.  The collaboration's technical coordinator is co-PI of \sDIPC. The two PIs of \sUPV\ are respectively the coordinator of the Data Acquisition (R. Esteve)  and the coordinator of the electronics front-end (V. Herrero). The software coordinator, J.A.~Hernando, is PI of \sUSC.  The two physics co-conveners including one of the co-PI are in the \sIFIC\ project. The other co-PI, J. Martín-Albo, is the convener of the Monte Carlo simulation and one of the leaders of the \NHD\ R\&D. Last, but not least, the PI of \sLSC\ is the director of the laboratory, C. Pe\~na-Garay.  

  
% \indent
% 
% The collaboration successfully built the first NEXT prototype operating in Spain (NEXT-DEMO, which is still in operation at the IFIC laboratory) and followed this with the first radiopure high pressure xenon chamber operating in the world, the \NEW\ detector, which has been taking data at Laboratorio Subterr\'aneo de Canfranc (LSC) until recently. \NEW\ has produced a host of results that have conclusively shown the potential of the technology, see section~\ref{sec.new} The construction and physics exploitation of \NEW\ was co-financed with grant FIS2014-53371-C4-1-R and an ERC Advanced Grant (AdG/ERC number 339787). The NEXT collaboration is currently finishing the construction of the \Next\ detector, described in section ~\ref{sec.next100} \Next\ is scheduled to start data taking in 2022. The construction of  \Next\  was co-financed with grant FIS2014-53371-C4-1-R and an ERC Advanced Grant (AdG/ERC number 339787). Both \NEW\ and \Next\ were also co-financed by the international NEXT collaboration and the LSC laboratory which contributed with infrastructures such as the lead and copper shields. LSC is also the owner of the xenon used by the experiment (100 kg of xenon enriched at 90\% in the \XE\ isotope and 100 kg of xenon 
% depleted on the \XE\ isotope). 
 
 %With a mass of 100 kg of xenon enriched at 90\% in the \XE\ isotope, \Next\ will be the largest high pressure xenon (HPXe) TPC operating in the World (a title currently held by \NEW) and could reach a sensitivity level similar to that of other major xenon experiments, such as EXO-200 and KamLAND-Zen. Even more importantly, the scientific exploitation of \Next\ is a major step towards developing the HPXe technology at the ton scale, where a discovery appears likely. 
 %that could place Spain at the forefront of the fields of particle and nuclear physics}. This ambitious goal requires strict coordination, not only among our groups, but across the entire international NEXT Collaboration.

 %\indent
 
 This proposal requests funds for the integration, commissioning and scientific exploitation of the \Next\ detector as well as for the R\&D needed or the next stage of the project, 
 %that we call \NHD (where the term HD comes from ``High Definition" and will be explained later) and 
 which involves the development of a ton-scale HPXe. The proposed activities require an intense collaboration between our groups and with the rest of the collaboration, as will be amply described in this document. 
 
 %The deliverables of the proposal are:{\bf i} physics results produced by \Next\ which include a precise measurement of the \bbtnu\ mode, a search for \bbonu\ events in \XE\ decays, with a sensitivity competitive with other leading experiments worldwide, and other searches, such as the search for \bbonu\ events in \XEX\ decays; {\bf ii} a precise measurement of the detector performance, including energy resolution, topological signal and radioactive budget. These are all needed as input for the design of the \NHD\ detector; {\bf iii} a Conceptual Design Report (CDR) that will present the results of our R\&D program to develop crucial components (such as the barrel fiber detector, the dense tracking planes and the associated electronics) for \NHD\ and {\bf iv} new infrastructures needed for \NHD\, such as a water-tank and the copper shield. IFIC, UPV, USC and DIPC, together with the international partners of the collaboration will work in deliverables {\bf i, ii} and to {\bf iii}. LSC will provide deliverable {\bf iv}, and will coordinate the radiopurity program for \NHD, which becomes crucial at the ton-scale and requires the extensive facilities available in the lab.
 
%  The involvement of LSC in the NEXT project has increased gradually, with the development of the program. For the \NEW\ stage, LSC contributed with several infrastructures, specifically the shielding (the detector was hosted inside a ``lead castle") and the radon free atmosphere. For the \Next\ stage, LSC has contributed with the copper shielding located inside the detector. In addition, LSC is the owner of the enriched xenon used by the experiment. 
 
% The next stage of the project, that we call \NHD (where the term HD comes from ``High Definition" and will be explained later) involves the construction, commissioning and operation of a ton-scale HPXe. For this stage of the project, the LSC enters as an active partner, with major responsibilities in several crucial components, including the new water tank and the massive copper shield needed for the apparatus, as well as the coordination of the radiopurity program, which becomes crucial at the ton-scale and requires the extensive facilities available in the lab. Like in the previous phases, LSC will be the owner of the enriched xenon. 
 
 %The groups presenting this proposal share major responsibilities. They will cooperate in the operation and data analysis of \Next\ and will jointly develop the R\&D needed for \NHD. %This necessary coordination has been formalized since the early stages of NEXT through the NEXT Project Management Plan (PMP), briefly described in Sec.~\ref{sec.pmp}.

% \indent
 

%The NEXT Collaboration was formed in 2008 by J.J.~G\'omez Cadenas. In 2011 a Conceptual Design Report (CDR) was published \cite{Alvarez:2011my}, choosing the electroluminescent (EL) technology proposed by D. Nygren as the amplification method and basing the tracking in arrays of SiPMs, as proposed by Gómez-Cadenas. In 2012, a Technical Design Report summary \cite{Alvarez:2012haa} further refined the main design ideas of the \Next\ detector. 

 %\indent
 
The current composition of the NEXT collaboration includes 20 institutions. Six institutions are from Spain:  Instituto de F\'isica Corpuscular (IFIC), Donostia International Physics Center (DIPC), U.~Polit\`ecnica de Val\`encia (UPV), U.~de Santiago de Compostela (USC), U.~A.~Madrid (UAM) and U.~Zaragoza (UZ). The international contingent includes three groups from Portugal: U.~Aveiro (UV), U.~Coimbra-LIBPhys and U.~Coimbra-LIP. Five institutions are from USA: Argonne National Laboratory (ANL), Fermi National Accelerator Laboratory (FNAL), Harvard University (HU), Iowa State University (ISU), U.~of Texas at Arlington (UTA). Two groups are from Israel, Ben-Gurion University (BGU), and Weizmann Institute of Science (WIS). And one group from Germany, Justus-Liebig University. The number of authors (in 2021) signing NEXT publications is 110. 

