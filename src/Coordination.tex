
 The scientists which participate in four of the five groups presenting this proposal form the core of the Spanish part of the NEXT collaboration, and have been working together for the last ten years. They have participated in the CONSOLIDER-INGENIO project CUP, in the coordinated project FIS2014 and in the coordinated project FPA2018. The collaboration successfully built the first NEXT prototype operating in Spain (NEXT-DEMO, which is still in operation at the IFIC laboratory) and followed this with the first radiopure high pressure xenon chamber operating in the world, the \NEW\ detector, which has been taking data at Laboratorio Subterr\'aneo de Canfranc (LSC) until recently. \NEW\ has produced a host of results that have conclusively shown the potential of the technology, see Sec.~\ref{ch.new}. We currently finishing the construction of the \Next\ detector, which is scheduled to start data taking in 2022. With a mass of 100 kg of xenon enriched at 90\% in the \XE\ isotope, \Next\ will be the largest high pressure xenon (HPXe) TPC operating in the World (a title currently held by \NEW) and could reach a sensitivity level similar to that of other major xenon experiments, such as EXO-200 and KamLAND-Zen. Even more importantly, the scientific exploitation of \Next\ is a major step towards developing the HPXe technology at the ton scale, where a discovery appears likely. 
 %that could place Spain at the forefront of the fields of particle and nuclear physics}. This ambitious goal requires strict coordination, not only among our groups, but across the entire international NEXT Collaboration.
 
 The fifth group participating in this proposal is integrated by LSC personnel. The involvement of LSC in the NEXT project has increased gradually, with the development of the project. For the \NEW\ stage, LSC contributed with several infrastructures, specifically the shielding (the detector was hosted inside a ``lead castle") and the radon free atmosphere. For the \Next\ stage, LSC has contributed with the copper shielding located inside the detector. In addition, LSC is the owner of the enriched xenon used by the experiment. 
 
 The next stage of the project, that we call \NHD (where the term HD comes from ``High Definition" and will be explained later) involves the construction, commissioning and operation of a ton-scale HPXe. For this stage of the project, the LSC enters as an active partner, with major responsibilities in several crucial components, including the new water tank and the massive copper shield needed for the apparatus, and the coordination of the radiopurity program, which becomes crucial at the ton-scale. Like in the previous phases, LSC will be the owner of the enriched xenon. 
 
 The groups presenting this proposal share major responsibilities. They will cooperate in the operation and data analysis of \Next\ and will jointly develop the R\&D needed for \NHD. This necessary coordination has been formalized since the early stages of NEXT through the NEXT Project Management Plan (PMP), briefly described in Sec.~\ref{sec.pmp}.


As an illustrative example, the founder and co-spokesperson of NEXT is  J.J.G\'omez Cadenas, who is also the coordinator of this project.  The collaboration's technical coordinator is F.~Monrabal, co-PI of \sDIPC. The two IPs of \sUPV\ are respectively the coordinator of the Data Acquistion (R. Esteve) and the coordinator of the electronics front-end (V. Herrero). The software coordinator, J.A.~Hernando, is IP of \sUSC.  The physics co-convener is P.~Novella, co-PI of \sIFIC. The other co-IP, J. Martín-Albo is the convener of the Monte Carlo simulation. Last, but not least, the IP of the \sLSC\ is the director of the laboratory, C. Pe~na-Garay.  

%The NEXT Collaboration was formed in 2008 by J.J.~G\'omez Cadenas. In 2011 a Conceptual Design Report (CDR) was published \cite{Alvarez:2011my}, choosing the electroluminescent (EL) technology proposed by D. Nygren as the amplification method and basing the tracking in arrays of SiPMs, as proposed by Gómez-Cadenas. In 2012, a Technical Design Report summary \cite{Alvarez:2012haa} further refined the main design ideas of the \Next\ detector. 

The current composition of the collaboration includes 20 institutions. Nine institutions are from Spain:  IFIC, DIPC,  UPV/EHU, CFM, U.~Girona (UG), U.~Polit\`ecnica de Val\`encia (UPV), U.~de Santiago de Compostela (USC), U.~A.~Madrid (UAM) and U.~Zaragoza (UZ). The international contingent include three groups from Portugal: U.~Aveiro (UV), U.~Coimbra-LIBPhys and U.~Coimbra-LIP. Seven institutions are from USA: Argonne National Laboratory (ANL), Fermi National Accelerator Laboratory (FNAL), Harvard University (HU), Iowa State University (ISU), U.~of Texas at Arlington (UTA), PNNL  and Argonne National Laboratory (ANL). Two group are from Israel, Ben-Gurion University (BGU), and Weizmann Institute of Science (WIS). And one group from Germany, Justus-Liebig University. The number of authors (in 2021) signing NEXT publications is 110. 

