\subsection{Objectives of the \sUPV\ subproject}
\label{sec.obj.upv}
\subsubsection*{Operation of \Next}

UPV has developed the DAQ, the front-end electronics for the energy and tracking planes, the online system and the slow controls (in collaboration with IFIC) for the \NEW\ and \Next\ detectors. Consequently, one major task of the group is   providing the maintenance of all the front end electronics modules and the DAQ modules. In addition, UPV will complete the development of SC, in collaboration with IFIC. 

%\indent
The commissioning and operation of the system will occur during the second half of 2022, UPV will be in charge to monitor the day-by-day operation of the detector DAQ, that is to ensure that the system works properly in terms of performance and data integrity. This includes the SCs related to the DAQ system that monitors the status of the EP, TP and TPC in terms of temperature and operational voltages. The slow-control can react automatically to emergencies such as over temperature or over current consumption that needs to be checked by an expert operator. Moreover, maintenance of the different modules, cleaning of cards and cable contacts over the time, or replacing of damaged modules, will be needed over the years of operation.


\subsubsection*{\HDEMO\ prototype}

%\indent
UPV will be in charge of developing the front-end electronics and DAQ for \HDEMO.
Following \HDEMO\  prototype philosophy, the idea is to use it as a test-bench also for front-end electronics and DAQ developments. The new front-end paradigm based on integrated solutions will make use of it to check and characterise ASIC prototypes and also to get the specifications for future ASIC in the front-end design cycle. UPV will produce, in collaboration with Harvard, a prototype of the DSP by the first half of the third year.  the pDSP will be equipped with a prototype of the new ASIC front-end electronics, deploying less channels than the final version but the same functionality. UPV will also contribute with a DAQ solution compatible with the new front-end architecture.

%As a backup solution, HD-DEMO prototype could be equipped with a modified version of NEXT-100 front-end electronics and DAQ. Since \HDEMO\ constraints related to DSP number of channels are less strict than those expected from \NHD, SiPM analog output signals can be sent to the outside of the vessel using standard feedthrough solutions and processed using COTS based frontend electronics. 

%Regarding slow control, NEXT-100 scheme can be applied since it is flexible and configurable enough to adapt to \HDEMO\ requirements. 


 
\subsubsection*{FEE and DAQ for \NHD}

%\indent
Due to the high number of channels in the DSP of \NHD\ an ASIC based architectures is the best solution to reduce the number of signals to be sent to DAQ from the inside of the vessel using feedthroughs. The proposed architecture for this \NHD\ frontend has a two level structure: analog frontend ({\bf AFE}) plus digital communications hub ({\bf DCH}) where a certain number of AF will be connected to a single CH:
\begin{itemize}
    \item {\bf Analog Frontend ASIC}: It will have two different sections: The first will be composed by analog electronics and an ADC converter for a certain number of channels (64 as an initial guess). The second one will be mainly a digital section with a FIFO system to deal with the data generated at the output of the channels ADCs and a "low speed" digital communications link to send the information out of the ASIC. Zero supression techniques might help to reduce required speed and alleviate power consumption.
    \item {\bf Digital Communication Hub ASIC}: It will deliver clock, trigger signaling and slow control data to a certain number of AF ASICs. This device will also receive data from those ASICs and implement a buffering stage to enhance data transmission. Data transmission will be carried out using a minimum number of copper or fiber high speed communication links.
\end{itemize}

%\indent 
Another important task included in the frontend design process will be the development of custom substrates, printed circuits boards and cabling solutions for the ASIC based scheme. Due to the tight radiopurity requirements this task requires special skills that have been learned in previous NEXT experiment stages.
The DAQ system must be adapted to the ASIC based frontend. The new architecture should focus in data transmission and bandwidth optimization since analog preprocessing and analog to digital conversion of the SiPM output signals, are carried out at the ASIC level. In order to improve DAQ performance and keep up with the latest technologies \sUPV\ group is involved in the CERN RD51 programme. Previous NEXT experiments have been successfully using DAQ electronics jointly developed by UPV and CERN under this collaboration. This development, named SRS for Scalable Readout System, is to receive an update for newer technologies and higher performance. UPV group will take part in this effort and use this new DAQ for \NHD.

%\indent 

%Along with the proposed main design-flow, UPV group has contacted Dr. Fischer at ZITI group from Heidelberg University who is in charge of the development of digital SiPM devices for physics applications. This type of devices could be an alternative to standard analog SiPM devices, however a careful test must be carried in order to check if they meet \NHD\ specifications.

%\indent 

\subsubsection*{Personnel in the Working Plan}
The working plan for \sUPV\ involves 5 FTEs (V. Herrero, R. Esteve, F. Toledo, F. Mora and F. Ballester). The responsibilities on the project of the existing personnel are:
\begin{itemize}[noitemsep,topsep=0pt,parsep=0pt,partopsep=0pt]
\item V. Herrero (senior analog microelectronic designer, Associate Professor at UPV): ASIC development coordinator, analog front end and ADC designer will be responsible of silicon design flow.
\item R. Esteve (senior digital microelectronic designer, Associate Professor at UPV): DAQ development coordinator and digital designer will be responsible for high speed data acquisition systems and FPGA firmware development.
\item F. Toledo (senior digital systems designer, Associate Professor at UPV): Digital systems (PCBs etc.) development coordinator and DAQ designer will be responsible for new DAQ hardware design and assessment.
\item F. Mora (senior digital systems designer, Full Professor at UPV): High speed communication links coordinator and DAQ designer will be responsible for DCH ASIC .
\item F. Ballester (senior digital systems designer, Associate Professor at UPV): NEXT-100 electronics operation and commissioning coordinator and responsible of future slow control hardware developments
\end{itemize}

In this project we request funding for a junior microlectronic engineer and a senior electronic system engineer. The former will work on the ASIC development including analog design for AFE and digital design and firmware for DCH. The latter will be devoted to several tasks: system developments (PCBs, cabling, test platforms) associated to ASIC; new DAQ architecture planning and sizing of its elements; \Next\ final commissioning along with its maintenance including new slow control hardware designs. 
%Regarding the request for a junior microelectronic engineer, from a realistic point of view, nowadays is difficult to find senior microelectronic engineers due to the outstanding job conditions in microelectronic design companies. UPV group will recruit a junior microelectronic engineer and train him/her in specific design skills to overcome this situation. 
\sUPV\ also requests a FPI fellow with a microelectronic profile in order to contribute to design tasks in the late ASIC development stages. Thanks to the training capabilities of the research team, an FPI student will have enough design skills by the end of the second year. The student will also contribute to prototypes test and characterisation.

