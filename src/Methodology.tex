\subsection{Methodology}



The specific objectives of all the subprojects are integrated in the NEXT Project Management Plan (PMP). The PMP has been used to organise the construction and operation of \NEW, and is currently managing the integration, commissioning and operation of \Next. 
The PMP has been essential for the construction of both apparatus. In particular, the \Next\  project drew from the experience gained while building \NEW, resulting in optimised procedures for the construction of the detector and the development of the electronics and software. The PMP for \NHD\ currently under development draws on the experience acquired building \Next\ and is an essential tool to manage the project.  
\indent

%\begin{figure}
%  \begin{center}
%    \includegraphics[width=0.91\textwidth]{img/pmp-n100.png}
%    \caption{A sketch of the NEXT PMP showing the main projects and \Next\ subprojects} 
%    \label{fig.pmps}
%  \end{center}
%\end{figure}

%\begin{figure}
%  \begin{center}
%    \includegraphics[width=1.0\textwidth]{img/pmp-nhd.png}
%    \caption{A sketch of the NEXT PMP showing the main projects and the main \NHD subprojects prior to construction. } 
%    \label{fig.pmp}
%  \end{center}
%\end{figure}

%The five groups involved in this proposal collaborate in the projects defined above:
%
%\begin{itemize}[noitemsep,topsep=0pt,parsep=0pt,partopsep=0pt]
%\item {\bf NEXT-100}: integration of the final detector is being led by DIPC and IFIC, with participation of the international collaboration. USC is in charge of the software development and maintenance. UPV is in charge of the electronics, DAQ and slow controls. IFIC, DIPC and USC collaborate in the physics (IFIC leads the development of physics tools and USC the calibration procedures). All the groups participate in the data taking and daily operations of the apparatus. 
%\item {\bf NEXT-HD R\&D}: the R\&D effort will focus in the \HDEMO\ apparatus. IFIC will lead the construction of the BFD prototype, UPV of the DSPs and associated electronics, DIPC will be in charge of the infrastructures (gas system and shielding), pressure vessel and TPC. The three groups will collaborate in the integration. USC will prepare the software, simulation and calibration procedures. 
%\item {\bf NEXT-HD R\&D}: The DSPs of the \NHD\ detector will have near 200,000 channels. This makes imperative using integrated front-end electronics. UPV will be leading the effort of developing an ASICs, for the DSPs, drawing on their experience with \Next\ front-end electronics. In addition they will develop an upgraded version of the existing DAQ. 
%\item {\bf NEXT software}:  NEXT software includes the offline reconstruction, online monitoring, automatic calibration software, Monte Carlo simulation and analysis tools. All the software components are fully operative for \NEW\ and \Next\ apparatus and will be upgraded for \NHD\ under the leadership of USC.  
%\end{itemize}
 

\indent

The PMP is under the direct supervision of NEXT executive spokesperson and coordinator of this proposal (G\'omez-Cadenas). He is assisted by a project manager (PM) who combines an extensive technical preparation with experience in logistics. DIPC has hired a dedicated person for this task (F. L\'opez-Guejo), who is currently serving as \Next\ PM. To ensure that the software for \NHD\ is developed also under controlled protocols, the collaboration has appointed a software coordinator (SC, J.A. Hernando, PI of the \sUSC\ project). 

\indent
 
The PMP defines and follows the progress of a set of Working Packages (WP), monitoring deliverables and deadlines and keeps track of invested resources including personnel. It also identifies potential show-stoppers and synergies (as well as possible conflicts) between the different projects and optimises the sharing of resources. 

\indent

The methodology of each WP includes: a) the definition of the associated tasks; b) the identification of the resources needed; c) the temporal organisation of the tasks; d) the definition of milestones and the deliverables associated to them; e) the relations with other WP. Each WP has a leader, who reports directly to the project managers. The progress of each WP is reviewed on a bi-weekly basis. Milestones and potential showstoppers are discussed, and the tracking charts updated if needed.

Coordination and monitoring of the various WP activities is ensured via a number of regular meetings within the NEXT Collaboration, run on a weekly or bi-weekly basis: These are the  {\em hardware} meeting, chaired by the technical coordinator, {\em software} meeting, chaired by software coordinator, and {\em analysis} meeting, chaired by the physics analysis conveners (PC, Michel Sorel and P. Novella from IFIC). The {\em project} meeting, chaired by the co-spokesperson (assisted by the PM), runs usually on a monthly basis, and involves the technical coordinator, software convener and physics conveners as well as the integration coordinator, the run coordinator and the leaders of the working packages. 

\indent

While it is not possible to capture all the details of the PMP\footnote{Detailed GANTT charts can be found in the NEXT web page, XXXX} in this document, we present a brief summary of the main projects in tables \ref{tab:schedule_n100}, \ref{tab:pmp_elec_nhd}, \ref{tab:pmp_hdemo}, \ref{tab:pmp_nhd_infra} and
\ref{tab:pmp_nhd}. Table  \ref{tab:schedule_n100} shows the schedule for the integration, commissioning and operation of \Next; 
table \ref{tab:pmp_elec_nhd} shows the schedule for the development of the electronics, table  \ref{tab:pmp_elec_nhd} details the R\&D for the \HDEMO\ prototype, and table \ref{tab:pmp_nhd_infra} shows the schedule for the development of the infrastructures for \NHD. All those activities will develop during the period
2022-2025, relevant for this project. For completeness, in table \ref{tab:pmp_nhd}, we show the foreseen schedule for the construction, integration and operation of \NHD.


\begin{table}[h!]
\begin{center}
\begin{tabular}{| l | c | c | c | c |}
\hline
Tasks & 2022 & 2023 & 2024 & 2025 \\
\hline
Integration  & Q1-Q2& -&-& -  \\
%\hline
Commissioning  & Q3-Q4 &-&- & -  \\
%\hline
Depleted xenon run &- & Q1-Q4 &- &-   \\
%\hline
Enriched xenon run  & -& - & Q1-Q4&  Q1-Q4 \\
\hline
\end{tabular}
\caption{Timetable for the integration, commissioning and operation of \Next.}
\label{tab:schedule_n100}
\end{center}
\end{table} 


\begin{table}[h!]
\begin{center}
\begin{tabular}{| l | c | c | c | c |}
\hline
Tasks & 2022 & 2023 & 2024 & 2025 \\
\hline
Prototype FEE  & Q1-Q2& -&-& -  \\
%\hline
Prototype DAQ  & Q1-Q2 &-&- & -  \\
%\hline
Production FEE \& DAQ for \HDEMO &- & Q1-Q4 &- &-   \\
%\hline
Final design FEE \& DAQ for \NHD  & -& - & Q1-Q4&  - \\
Production FEE \& DAQ for \NHD  & -& - & &  Q1-Q4 \\
\hline
\end{tabular}
\caption{Timetable for the development of \NHD\ electronics.}
\label{tab:pmp_elec_nhd}
\end{center}
\end{table} 

\begin{table}[h!]
\begin{center}
\begin{tabular}{| l | c | c | c | c |}
\hline
Tasks & 2022 & 2023 & 2024 & 2025 \\
\hline
R\&D pBFD  & Q1-Q2& -&-& -  \\
%\hline
R\&D pDSP  & Q1-Q2 &-&- & -  \\
R\&D pTPC, Vessel, HVFT \& Grids  & Q1-Q2 &-&- & -  \\
%\hline
\HDEMO\ construction  &- & Q1-Q4 &- &-   \\
%\hline
\HDEMO\ integration  & -& - & Q1-Q2&  - \\
\HDEMO\ commissioning  & -& - & Q3-Q4&  - \\
\HDEMO\ run  & -& - & -&  Q1-Q4 \\
\hline
\end{tabular}
\caption{Timetable for the development of the \HDEMO\ prototype.}
\label{tab:pmp_hdemo}
\end{center}
\end{table} 


\begin{table}[h!]
\begin{center}
\begin{tabular}{| l | c | c | c | c |}
\hline
Tasks & 2022 & 2023 & 2024 & 2025 \\
\hline
Selection \& screening of copper for ICS  & Q1-Q2& -&-& -  \\
%\hline
Design of Water Tank  & Q1-Q2 &-&- & -  \\
R\&D pTPC, Vessel, HVFD \& Grids  & Q1-Q2 &-&- & -  \\
%\hline
Procurement of copper  &- & Q1-Q4 &- &-   \\
%\hline
Machining of copper  & -& - & Q1-Q2&  - \\
Conditioning of \HDEMO\ experimental area  & -& - & Q1-Q2&  - \\
Construction of Water Tank  & -& - & Q3-Q4&  - \\
Upgrade of gas system  & -& - & -&  Q1-Q2 \\
\hline
\end{tabular}
\caption{Timetable for the construction of \NHD\ infrastructures.}
\label{tab:pmp_nhd_infra}
\end{center}
\end{table} 

\begin{table}[h!]
\begin{center}
\begin{tabular}{| l | c | c | c | c |}
\hline
Tasks & 2026 & 2027 & 2028 & 2029 \\
\hline
Construction of BFD, DSP and TPC  & Q1-Q4& -&-& -  \\
%\hline
Construction of PV, HVFT and Grids  & Q1-Q4 &-&- & -  \\
Production of electronics modules  & Q1-Q2 &-&- & -  \\
%\hline
Integration with ICS and WT  &- & Q1-Q2 &- &-   \\
Commissioning  &- & Q3-Q4 &- &-   \\
%\hline
Physics run  & -& - & Q1-Q4& Q1-Q4 \\
\hline
\end{tabular}
\caption{Timetable for the construction of \NHD.}
\label{tab:pmp_nhd}
\end{center}
\end{table} 






