%\subsection{General objectives and match to the National Programme for research aimed at the Challenges of Society}

\subsection*{General objectives}

The four main objectives of this proposal are:
\begin{enumerate}
\item {\bf The scientific exploitation of the \Next\ detector}. This includes the commissioning of the apparatus and the operation through the physics run, initially foreseen for three years. The deliverables are: {\bf i,} physics results produced by \Next\ which include a precise measurement of the \bbtnu\ mode, a search for \bbonu\ events in \XE\ decays, with a sensitivity competitive with other leading experiments worldwide, and other searches, such as the search for \bbonu\ events in \XEX\ decays; {\bf ii,} a precise measurement of the detector performance, including energy resolution, topological signal and radioactive budget. This deliverable is of uppermost importance for the second objective. 
\item {\bf The construction, commissioning and operation of \HDEMO}, as a platform to test the solutions to be implemented in \NHD.  
\item {\bf Selection and screening of materials for \NHD}. As discussed in the previous section, the radioactive budget of \NHD\ must be carefully controlled. This requires an extensive campaign of material selection and screening which will include the copper for the shielding and TPC, the steel (or titanium) for the vessel and grids, the PTFE for the light tube, fibres, SiPM circuit boards, resistors, etc. 
\item {\bf Preparation of the infrastructures for \NHD}. The operation of \NHD\ will require substantial new infrastructures at the LSC, including: {\bf i,} a new water tank to veto muons, {\bf ii,} an upgrade of the gas system, {\bf iii,} the preparation of the inner copper shield of the detector. 
\end{enumerate}

\indent

The scientific exploitation of the \Next\ detector (objective {\bf 1}) will be responsibility of DIPC, IFIC,  USC and UPV. The four institutions will cooperate in the operation of the detector itself (which requires day-by-day monitoring, slow-controls and shifts). DIPC will be in charge of run coordination, USC will lead the calibration of the detector and IFIC will provide analysis tools. UPV will be in charge of slow-controls and monitoring. All the institutions will collaborate in the physics analysis.  

\indent

The R\&D leading to a conceptual design report (CDR) (objective {\bf 2}) will be responsibility of DIPC, IFIC, USC and UPV. 
During the first two years of the project IFIC will build the pBFD (together with WIS and BGU from Israel), UPV will build the pDSP (in collaboration with Harvard) and a prototype of the front-end electronics and DAQ, DIPC will build the pTPC, the pressure vessel and the gas system, and USC will prepare the reconstruction software as well as the Monte Carlo simulation of the detector. The full system will be assembled in the first part of the third year, and operated starting in Q2 of the third year.  

The material selection and screening campaign (SSC) for \NHD\ (objective {\bf 3})  will be coordinated by LSC, in close collaboration with the NEXT groups. In particular, IFIC will cooperate with LSC for the SSC of the BFD materials as well as the light tube, DIPC for the SSC concerning the cooper and the materials for the vessel, grids, resistors and high-voltage feedthroughs, USC will be in charge of compiling the SSC data base and computing from it the \NHD\ background model, and UPV will be cooperate with LSC in the SSC of the electronics components  kapton boards and SiPMs. 

The preparation of the infrastructures for \NHD (objective {\bf 4})  will be lead by LSC, with the participation of the NEXT institutes, in particular DIPC, who will be in charge of integrating the \NHD\ detector with the shielding and gas system infrastructures. 
  