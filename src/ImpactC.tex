
%\subsection{Scientific and technological impact}

Establishing if the neutrino is a Majorana particle is one of the most important questions yet unanswered in particle physics. Majorana neutrinos, together with CP violation are a necessary ingredient of leptogenesis, an thus the only known way to explain the cosmic asymmetry between matter and antimatter. Majorana neutrinos are also handy to explain the smallness of neutrino masses through the see-saw mechanism. It follows that {\em demonstrating} that neutrinos are indeed their own antiparticles, would be a major scientific achievements.

The only practical way to search for Majorana neutrinos is through the study of \bbonu\ processes. Observation of a \bbonu\ decay implies necessarily that the neutrino is its own antiparticle. On the contrary, the no-observation of the process does not rule out Majorana neutrinos, given the existence of cancelations in the decay probability for very light masses. 

Neutrinoless double beta decay experiments need to deal with the fact that the lifetimes of \bbonu\ processes are known to be very long (in excess of $10^{26}$ year, and very likely at least an order of magnitude larger) and with the fact that natural radioactivity and cosmogenic backgrounds are many orders of magnitude those of the putative signal. Consequently, they face the challenge to design apparatus deploying a large mass of the target isotope, while at the same time devising experimental techniques to reduce the huge background to virtually zero. 

NEXT is one of the major experimental programs currently being developed, as evidenced in the recent APPEC report \cite{Giuliani:2019uno}. It was also one of the four experiments presented in the recent {\it North America- Europe Workshop on Future Double Beta Decay}\footnote{\url{https://agenda.infn.it/event/27143/}}. The conclusions of the workshop highlighted the high potencial of the NEXT technology, in particular due to Barium Tagging. On the other hand, the program presented in this proposal is a necessary step towards the development of a ton-scale detector able to reach the target sensitivity of the next generation of experiments
($10^{27}$ year) by extending and improving the technology developed by NEXT so far. The Barium Program R\&D, financed both in the US and in Europe (in particular through Synergy Grant "BOLD"), will be developed in parallel with the R\&D and construction of a ton-scale detector. Both programs can converge in a few years, resulting in a ton-scale detector with barium tagging, able to search for \bbonu\ events with virtually zero background, and thus capable to push the sensitivity to  $10^{28}$ year and beyond, with a very large potential for a scientific discovery.

The development of the NEXT experiment places Spain among the World leaders in a major scientific program. The development of the HPXe technology has only been possible thanks to the acquisition of the scientific know-how and the logistics to build a large scientific instrument. Last but not least, the the project has been able to involve the industrial sector at many different levels. Remarkably, the experiment ---and Spain, as host--- competes with other projects, such as LEGEND, CUPID or nEXO, which have a much longer development and can mobilise, a priori, larger resources. This is so because NEXT has managed to consolidate a full value chain, including a critical mass of national academic and industrial partners and singular facilities such as the LSC.  

This has important implications regarding the impact of the project beyond the scientific boundaries:
\begin{itemize}[noitemsep,topsep=0pt,parsep=0pt,partopsep=0pt]
\item The construction of NEXT detectors requires some national industrial partners to go beyond their own current capacities. The development of state-of-the-art technologies contributes to augment their international visibility and strengthens their showcase portfolio, helping them to diversify into markets such as those of large scientific instruments. 
\item  NEXT provides support to the aforementioned industrial partners both by sharing know-how and by collaborating with them in attracting funding. Sources of this funding can be both public competitive calls and/or private investment. Such focused investment with clear and ambitious technical targets optimises the use of funds.  
\item The successful development of NEXT will demonstrate that LSC can be a reliable long term partner, with a clear role within the global network of underground laboratories. While LSC cannot compete in size and depth with some other underground facilities ---like the LNGS, at Gran Sasso, in Italy--- it is strategically well placed within European soil, has easy horizontal access and can adapt its resources in a flexible and reliable manner. This will rise the international profile of this national singular facility and attract new relevant scientific projects in fields such as quantum technologies, biology and particle physics. 
\item On the individual scale, it is worth stressing that NEXT unusual environment provides many young scientist, engineers and technologist with rather unique work experience. The interaction with companies eases the transfer of these professionals back and forth  between the academic and industrial environments, a process which is still a major barrier form many. 
\end{itemize}

The technological output of the project deserves a detailed look on its own. A complex instrument such as the HPXe detector implements a large number of new technologies, many of them pushed to unprecedented limits of performance. While is it possible that applications of some of these technologies remains still unforeseen, in some other cases we have already identified domains of application with important economical and societal impact: 

\begin{itemize}[noitemsep,topsep=0pt,parsep=0pt,partopsep=0pt]
\item The HPXe detector integrates technology used to detect and characterise the light events taking place within the xenon contained inside the detector. This technology can be implemented in Positron Emission Tomography instruments that will use xenon (in this case in liquid form, for increased density) instead of scintillating crystals such as LYSO.  This concept is being explored by ERC Starting Grant PETALO (granted to Associate IKERBASQUE Professor P. Ferrario, one of the NEXT leading researchers) and by a Competitive Purchase (Compra P\'ublica Innovadora), launched by Generalitat Valenciana.

\item The R\&D associated to barium-tagging is opening a new inter-disciplinary field. Our approach incorporates elements of organic chemistry, atomic physics, nuclear physics, particle physics, laser-matter interactions, and photonics. It involves the technology of \HPXeEL\ chambers, ion sources, magnetic-traps, and lasers (both visible and IR). There is a clear potential to develop experimental techniques associated to our experiments that go beyond our particular needs and have wide applications. This has been acknowledged by the ERC Synergy Grant awarded to G\'omez-Cadenas, Coss\'io and Guenette. can be used for detection of other molecules, or elements such as viruses, in extremely low concentrations. 
\end{itemize}
