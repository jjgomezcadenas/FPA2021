\subsection{\label{subsubsec:training}Training plan}

\indent 

This coordinated project requests 3 Ph.D. fellowships (\sDIPC, \sIFIC\ and \sUSC). The students will be enrolled in the Nuclear and Particle Physics doctorate programs at University of Pais Vasco, University of Valencia and University of Santiago de Compostela. As part of their training, students will attend at least one international school. Students will also spend a fraction of their time in international institutions participating in NEXT: Harvard, Fermilab, Texas University, Ben Gurion University, Weizmann Institute of Science and Coimbra University.

\indent 

This coordinated project is highly multi-disciplinary, offering the students the capability of cross-training and of developing skills in a variety of areas, given the wide scope of NEXT R\&D, the sophisticated data reconstruction with algorithms suitable for medical physics, development of Deep Neural Networks for pattern recognition, and state-of-the-art instrumentation. Students in this coordinated project will rotate between different areas of work (and groups) to acquire a wide background, on hardware and software, before settling onto a specific topic.

\subsection{Supervision of Ph.D. theses and career development of former Ph.D. students}

The scientific exploitation of \Next\ offers a great opportunity for graduate students and post-docs, who typically take leading roles in important parts of the experiment. The NEXT Collaboration follows a publication model in which the main authors of the analysis are the first authors (as opposed to the general trend in the field of particle physics, which uses alphabetical order). The goal of this policy is to facilitate the visibility of graduate students and young post-docs, and to encourage them to take leading roles in the development of the experiment and the analysis of the data. Student and post-docs regularly present their results in the most relevant conferences of the field, and apply to competitive calls such as RyC and ERC. Ferrario recently obtained a StG/ERC grant for building a medical PET, PETALO.

\subsubsection*{\sDIPC\ subproject}
The COORD subproject includes three senior physicists, G\'omez-Cadenas, Ferrario and Monrabal. G\'omez-Cadenas has advised (or co-advised) a total of 15 students.  Ferrario is advising one student in PETALO (Romo), and co-advising one student in NEXT (Benlloch). Monrabal is advising one student in NEXT (Felkai) and will co-advise one student in \Bapp-tagging R\&D (Mart\'{i}nez).

As explained in the scientific project, COORD proposes at least one additional student who will share his or her time between NEXT analysis and \Bapp-tagging R\&D. In addition, Ferrario will advise at least one student on PETALO (funded by ERC).

\subsubsection*{\sIFIC\ subproject}
Sorel has co-supervised two Ph.D. theses: Tornero ({\it ``Study of neutral pion production via neutrino-induced, charged-current interactions in the K2K SciBar detector''}, 2008) and Catal\`a ({\it ``Measurement of neutrino induced charged current neutral pion production cross section at SciBooNE''}, 2014). He is currently co-advising two Ph.D. students, both of them in NEXT (Palmeiro, Us\'on). Novella is co-supervising one Ph.D. student in NEXT (Us\'on) and another one in T2K (Antonova). L\'opez-March is co-advising one Ph.D. student in NEXT (Felkai). The NEXT-IFIC group is participating in EU H2020 grants focused on training ({\it ``Elusives''}, H2020-MSCA-ITN-2015) and staff exchange ({\it ``InvisiblesPlus''}, H2020-MSCA- RISE-2015-690575), to the benefit of current and prospective students in the group.

\sIFIC\ requests one Ph.D. fellowship covering both NEXT R\&D activities (under the guidance of L\'opez-March), and \NEW\ and \Next\ data analysis activities (under the supervision of Sorel and Novella).

\subsubsection*{\sUSC\ subproject}

Hernando has co-supervised two Ph.D. theses in the LHCb experiment: Mart\'{i}nez-Santos ({\it ``Study of the very rare decay $B_s\to\mu^+\mu^-$ in LHCb''}, 2010) and Cid Vidal ({\it ``Search for the rare decays $B^0_{(s)}\to\mu^+\mu^-$ and $K^0_S\to\mu^+\mu^-$ with 1~fb$^{-1}$ at LHCb''}, 2012), on one of the most important results of LHC run-I. He supevised one Ph.D. thesis in NEXT: Mart\'{i}nez-Lema ({\it ``Calibration and energy resolution of NEXT-White"}, 2018). Mart\'{i}nez-Santos was awarded the Young Experimental Physicist Prize by the European Physical Society (EPS) in 2013 and obtained in 2015 a StG/ERC grant. He is now Scientific Staff at IGFAE. Cid Vidal has been CERN fellow and is now at IGFAE with a RyC fellowship. Hernando is currently co-advising two Ph.D. students, both of them in NEXT (Palmeiro, D\'{i}az-Lopez).

\sUSC\ requests one Ph.D. fellowship to work in the areas of NEXT data analysis (with Hernando, Renner) and gas mixtures R\&D (with Gonz\'alez).
