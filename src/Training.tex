\subsection{\label{subsubsec:training}Training plan}

\indent 

This coordinated project requests 4 Ph.D. fellowships (\sDIPC, \sIFIC, \sUSC\ and \sUPV). The students will be enrolled in the Nuclear and Particle Physics doctorate programs at University of Pais Vasco, University of Valencia and University of Santiago de Compostela, and in the Electronics Engineering doctorate program at Universitat Politecnica de Valencia. As part of their training, students will attend at least one international school. Students will also spend a fraction of their time in international institutions participating in NEXT: Harvard, Fermilab, Texas University, Ben Gurion University, Weizmann Institute of Science and Coimbra University.

This coordinated project is highly multi-disciplinary, offering the students the capability of cross-training and of developing skills in a variety of areas, given the wide scope of NEXT R\&D, the sophisticated data reconstruction with algorithms suitable for medical physics, development of Deep Neural Networks for pattern recognition, and state-of-the-art instrumentation. Students in this coordinated project will rotate between different areas of work (and groups) to acquire a wide background, on hardware and software, before settling onto a specific topic.

\subsection{Supervision of Ph.D. theses and career development of former Ph.D. students}

The scientific exploitation of \Next\ offers a great opportunity for graduate students and post-docs, who typically take leading roles in important parts of the experiment. The NEXT Collaboration follows a publication model in which the main authors of the publications are the first authors (as opposed to the general trend in the field of particle physics, which uses alphabetical order). The goal of this policy is to facilitate the visibility of graduate students and young post-docs, and to encourage them to take leading roles in the development of the experiment and the analysis of the data. Students and post-docs regularly present their results in the most relevant conferences of the field, and apply to competitive calls such as RyC and ERC. 
%Ferrario recently obtained a StG/ERC grant for building a medical PET, PETALO.

\subsubsection*{\sDIPC\ subproject}
The COORD subproject includes three senior physicists, G\'omez-Cadenas, Ferrario and Monrabal. G\'omez-Cadenas has advised (or co-advised) a total of 15 students, and currently co-supervises two students (Romeo, Mart\'inez).  Ferrario is advising one (Romo). Monrabal is co-advising three students in NEXT (Mart\'{i}nez, Larizgoitia, Herrero).

%As explained in the scientific project, COORD proposes at least one additional student who will share his or her time between NEXT analysis and \Bapp-tagging R\&D. In addition, Ferrario will advise at least one student on PETALO (funded by ERC).

\subsubsection*{\sIFIC\ subproject}
M. Sorel has co-supervised two Ph.D. theses. These are that of
A. Tornero 
%({\it ``Study of neutral pion production via neutrino-induced, charged-current interactions in the K2K SciBar detector''}, 2008), 
now Radio-physicist at Hospital
Universitario Dr. Negrin, and that of J. Catal\`a 
%({\it ``Measurement of neutrino induced charged current neutral pion production cross section at SciBooNE''}, 2014), 
now Data Scientist at Blue Trail Software. He is currently co-advising four Ph.D. students, three of them in NEXT (M. Mart\'{i}nez, B. Palmeiro, A. Us\'on). P. Novella is co-supervising one Ph.D. student in NEXT (A. Us\'on). J. Mart\'in-Albo has co-supervised one PhD thesis in NEXT-White (M. Nebot, 
%{\it Calibration and background model of the NEW detector}, 2017, 
now Postdoctoral Research Associate at University of Edinburgh). 
%and is currently supervising another Ph.D. student in working on the DUNE experiment (P. Amedo). 
N. L\'opez-March is co-advising one Ph.D. student in NEXT (R. Felkai). The NEXT-IFIC group is participating in the European H2020 Innovative Training Network (ITN) "HIDDeN", to the benefit of current and prospective students in the group.

%\sIFIC\ requests one Ph.D. fellowship to work on the data taking and \bb\ analysis activities in \Next\ (under the supervision of P. Novella and M. Sorel). The student will have the opportunity to contribute also to the NEXT-HD R\&D program at IFIC.


\subsubsection*{\sUPV\ subproject}
V. Herrero is co-advising one Ph.D. student (V. Alvarez) working in the NEXT collaboration. R. Esteve has supervised two Ph.D. theses (C.A. Marín, 
%({\it ``PADRE pixel read-out architecture for Monolithic Active Pixel Sensor for the new ALICE Inner Tracking System in TowerJazz 180 nm technolog''}),  
E.J. García) 
%({\it ``Novel Front-end Electronics for Time Projection Chamber Detectors''}) 
and co-supervised one Ph.D. thesis (J.M. Monzó). 
%({\it ``Estudio e implementación de algoritmos digitales para la mejora de la resolución temporal en sistemas de tomografía por emisión de positrones''}). 
F. Mora has supervised three Ph.D. theses (J.J. García-Garrigos,
 %({\it ``Development of the Beam Position Monitors for the Diagnostics of the Test Beam Line in the CTF3 at CERN''}), 
 C.A. Luján, 
% ({\it ``Adquisición y Procesamiento Digital de Imátenes para la Obtención de la Trayectoria de los Vectores de Posición del Camarón y la Jaiba''}), 
 J.F. Toledo)
 %({\it ``Study and design of the readout unit. Module for the LHCb experiment''}) 
 and co-supervised one thesis (S. Coll).
 % ({\it ``A strategy for efficient and scalable collective communication in the quadrics network''}). 
  F.J. Ballester has co-supervised three Ph.D. theses (M.A. Acosta,
  % ({\it ``Sistema de Alerta temprana para la predicción del nivel de peligrosidad en inundaciones pluviales repentinas''}), 
  J.A. Canals, 
  %({\it ``Análisis y desarrollo de nuevas arquitecturas eficientes VLSI para la implementación de algoritmos basados en "Fast Search Block Matching''}), 
  A. Mora). 
  %({\it ``Estudio de Arquitecturas VLSI de la etapa de predicción de la compensación de movimiento, para compresión de imágenes y video con Algoritmos full-search. Aplicación al estándar H.264/AVC''}). 
  J.F. Toledo has supervised one Ph.D. thesis (J. Rodríguez). 
  %({\it ``Study and design of the front-end and readout electronics for the tracking plane in the NEXT experiment''}) 


\subsubsection*{\sUSC\ subproject}

Hernando has co-supervised two Ph.D. theses in the LHCb experiment (Mart\'{i}nez-Santos,
% ({\it ``Study of the very rare decay $B_s\to\mu^+\mu^-$ in LHCb''}, 2010) 
Cid Vidal). Mart\'{i}nez-Santos was awarded the Young Experimental Physicist Prize by the European Physical Society (EPS) in 2013 and obtained in 2015 a StG/ERC grant. He is now Scientific Staff at IGFAE. Cid Vidal has been CERN fellow and is now at IGFAE with a RyC fellowship.
%({\it ``Search for the rare decays $B^0_{(s)}\to\mu^+\mu^-$ and $K^0_S\to\mu^+\mu^-$ with 1~fb$^{-1}$ at LHCb''}, 2012), on one of the most important results of LHC run-I. 
Hernando has also supervised one Ph.D. thesis in NEXT, that of Mart\'{i}nez-Lema, currently a post-doc at Weizmann Institute of Science, in Israel. He is 
% ({\it ``Calibration and energy resolution of NEXT-White"}, 2018).  Hernando is 
currently co-advising three Ph.D. students, all of them in NEXT (G. D\'{i}az, C. Hervés, M. Pérez).

%\sUSC\ requests one Ph.D. fellowship.