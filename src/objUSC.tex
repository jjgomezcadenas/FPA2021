\subsection{Objectives of the \sUSC\ subproject}
\label{sec.obj.usc}

\indent


USC takes three main responsibilities within this project: {\bf i)} calibration of the \Next\ detector, {\bf ii)} coordination, maintenance and development of the \Next\ software, {\bf iii)} development of the simulation and reconstruction software for \HDEMO. USC will also contribute to the \Next\ operation and data analysis, as it was done with \NEW.

\subsubsection{Calibration of the \Next\ detector.}

The experience with \NEW\ has demonstrated that the calibration with radioactive sources, \KR, \CS, \TL, is essential to estimate the energy scale and to obtain the required energy resolution. The daily calibration with \KR, which continuously flows within the gas inside the chamber, allows us to monitor the lifetime evolution, and thanks to its point-like energy deposit (41.5 keV), to map the detector response in the active volume. \CS\ and \TL\ external sources  provide specific candles to estimate the energy scale and resolution. In addition, the double-escape peak of the \TL\ gamma of 2.6 MeV allows us to estimate the power of the topological discrimination. USC has been responsible of the calibration and monitoring of \NEW. These tasks will be again essential in \Next. 

\Next\ will be calibrated with ~4 million \KR\ triggers per day and with specific \CS\ and \TL\ runs every month. USC will be responsible of adapting, improving, and operating the calibration procedures of \Next. This activity will be done by the calibration manager, a role that will be played by senior scientist J. Renner and one FPI fellow, requested in this project. 

\subsubsection{Coordination of the \Next\ software}

The NEXT collaboration has pioneered the use of novel frameworks based on functional programming. Special attention has been devoted to the development of  robust and reusable code in order to facilitate the data exploitation by ``users" (\ie, physicists analysing data). The supervision and planning of the software project, as well as the maintenance, adaptation and improvement of the collaboration common code have been the responsibility of USC during \NEW\ operation. USC will also be in charge of the software project (the PI of \sUSC\ is the software coordinator of the collaboration).  

\Next\ software will require extensive software development, including: {\bf i)}  adaptation of the calibration code to \Next, {\bf ii)}  update and critical revision of the reconstruction algorithms, and {\bf iii)}  the inclusion of new tracking reconstruction and particle identification algorithms based on convolution techniques, image processing or Neural Networks. There has been a large progress in the last years in our understanding of the tracking reconstruction with \NEW\ data, and this progress will be adapted and improved for \NEXT\ and \NHD.

All the above requires a dedicated physicist-software developer, who will fill the role of software technical manager, ({\bf STM}). For this project we request a three-year position to fill this crucial role. 

\subsubsection{Simulation of \HDEMO\ prototype}

%%USC will participate, together with IFICThe simulation of theUSC, in particular G. Diaz (student), in collaboration with IFIC (J. Marti), has implemented the detailed GEANT4 simulation of \New\ and \Next, and the development of simulation algorithms to avoid the follow-up of the large number of photons produced in the full simulation. The simulation the NEXT detectors and demonstrators is crucial to improve its design and to understand its performance. 

USC will be in charge of the Monte Carlo simulation and reconstruction software for \HDEMO. Hernando and Renner will supervise one graduate student to work on the simulation, while the STM will supervise another graduate student to adapt the NEXT reconstruction code to \HDEMO. 

\subsubsection*{Personnel in the Working Plan}

The working plan for \sUSC\ involves 2 FTEs (J.A. Hernando, and J. Renner) and 3 Ph.D. students (G. Díaz, C. Hervés and M. Pérez). One of the students (D\'iaz) will defend his Ph.D. thesis in about one year, while the other two will work in the tasks described above. 

%\begin{itemize}[noitemsep,topsep=0pt,parsep=0pt,partopsep=0pt]
%\item J.A. Hernando is the software coordinator of NEXT. He is responsible of the organisation an planing of the collaboration software in close contact with the analysis coordinator and the experiment management.
%
%\item J. Renner, he is the responsible of the calibration of \New, and supervises a Ph.D. Student (C. Hervés)
%
%\item G. Díaz will finish his Ph.D during the duration of this project. His Ph.D. thesis is about the simulation and performance of \Next. He will participate in the initial parts of the \HDEMO\ simulation.
%
%
%\end{itemize}
%
%J. Renner will be financed by external sources. G. Díaz has a grant by Xunta de Galicia. C. Hervés, M. Pérez will be financed by external sources but they have a large probability to obtain a Ph.D. grant by la Xunta de Galicia.

This project requires funding for a 3 year position to fill the crucial role of STM. 
%In this project we {\bf request a postdoc} who will be the {\bf technical software coordinator} of the collaboration. The assigned tasks are:
%
%\begin{enumerate}
%    \item to be responsible of the technical code integration, review of the software contributions and develop common code for the collaboration. 
%    
%    \item  develop the \HDEMO\ simulation, and supervise a student (M. Pérez). 
%\end{enumerate}
%
We also request a FPI fellow, given the demonstrated training capabilities of the USC, and the large training potential of the project.