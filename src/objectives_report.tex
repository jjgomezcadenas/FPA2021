\subsection*{General objectives}

The five main objectives of this proposal are:
\begin{enumerate}
\item {\bf The scientific exploitation of the \Next\ detector}. This includes the integration \& commissioning of the apparatus and the operation through the physics run. The deliverables are: {\bf i),} physics results produced by \Next\ which include a precise measurement of the \bbtnu\ mode, a search for \bbonu\ events in \XE\ decays, with a sensitivity competitive with other leading experiments worldwide, and other searches, such as the search for \bbonu\ events in \XEX\ decays; {\bf ii,)} a precise measurement of the detector performance, including energy resolution, topological signal and radioactive budget. This deliverable is of uppermost importance for the second objective. 
\item {\bf The construction, commissioning and operation of \HDEMO}, as a platform to test the solutions to be implemented in \NHD.  
\item {\bf Selection and screening of materials for \NHD}. As discussed in the previous section, the radioactive budget of \NHD\ must be carefully controlled. This requires an extensive campaign of material selection and screening which will include the copper for the shielding and TPC, the steel (or titanium) for the vessel and grids, the PTFE for the light tube, fibres, SiPM circuit boards, resistors, etc. 
\item {\bf Preparation of the infrastructures for \Next\ and \NHD}. As previously discussed, the operation of \Next\ requires the addition of a muon veto. On the other hand, the operation of \NHD\ will require substantial new infrastructures at the LSC, including: {\bf i),} a new muon veto adapted to the much more stringent background suppression needs of ton scale, which impose the use of an instrumented water tank, {\bf ii),} an upgrade of the gas system, {\bf iii),} the preparation of the inner copper shield of the detector. 

\item {\bf Integrated electronics for the SiPMs}. Due to the high channel density in the DSP, the only feasible solution for the frontend is to place as much functionality inside the vessel as possible so that a minimum number of outputs are sent to the outside using feedthroughs. This fact implies a trade-off with power consumption, data output bandwidth and frontend complexity. The design of an ASIC should offer the best solution to solve this trade-off and provide a full custom solution adapted to the frontend requirements. The design flow of a complex ASIC takes a considerable amount of time and specific resources though the expected benefits greatly outweigh the potential drawbacks.
\end{enumerate}

%\indent

The scientific exploitation of the \Next\ detector (objective {\bf 1}) will be responsibility of DIPC, IFIC,  USC and UPV (with the participation of all the other institutes in the collaboration). The four institutions will cooperate in the operation of the detector itself (which requires day-by-day monitoring, slow-controls and shifts). DIPC will be in charge of run coordination, USC will lead the calibration of the detector and IFIC will provide analysis tools. UPV and IFIC will be in charge of slow-controls and online monitoring. All the institutions will collaborate in the physics analysis.  

%\indent

The \HDEMO\ detector (objective {\bf 2}) will be built and operated through the collaboration of DIPC, IFIC, USC and UPV (with the help of international partners). 
During the first two years of the project IFIC will build the pBFD (together with WIS and BGU from Israel), UPV will build the pDSP (in collaboration with Harvard) and a prototype of the front-end electronics and DAQ, DIPC will build the pTPC, the pressure vessel and the gas system, and USC will prepare the reconstruction software as well as the Monte Carlo simulation of the detector. 

The material selection and screening campaign for \NHD\ (objective {\bf 3})  will be coordinated by LSC, in close collaboration with the NEXT groups. 

The preparation of the infrastructures for \NHD\ (objective {\bf 4})  will be led by LSC, with the participation of the other groups, in particular DIPC and IFIC, who will be in charge of integrating the \NHD\ detector with the shielding and gas system infrastructures.

Detailed schedules are presented in the section on Methodology.  
  