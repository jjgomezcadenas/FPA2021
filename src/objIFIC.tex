\subsection{Objectives of the \sIFIC\ subproject}
\label{sec.obj.ific}


\indent
IFIC takes three main responsibilities within this project: {\bf i}: integration, commissioning and operation of \Next\ (in collaboration with DIPC); {\bf ii}: the scientific exploitation of \Next; and {\bf iii}: R\&D towards the construction of the BFD detector and production of the pBFD prototype. In addition, IFIC will collaborate with DIPC and LSC in the definition of the \NHD\ interfaces and integration.

%\subsubsection*{Construction of the Muon Veto}
%The 14 modules of the system will be built and tested at IFIC during 2022. As the muon veto is only relevant during the low-background data taking of \Next, the installation will take place in early 2023, after the commissioning and first calibration campaign. The assembly and testing of the modules will be carried out by a post-doc (PD3), with the support of M. Sorel and P. Novella (CSIC Staff Scientists), S. C\'arcel (senior mechanical engineer) and a senior mechanical engineer (E1). In this project, we request funding to hire PD3 and E1 during 3 years.  

\subsubsection*{Integration and operation of \Next}

In the first half of 2022, the IFIC will contribute to the integration of \Next\ with S. C\'arcel (senior mechanical engineer), who co-coordinates the integration of the system with J. Torrent (DIPC). 
%S. C\'arcel has developed an extensive expertise by being in charge of the mechanics of NEXT-White pressure vessel. 
During the operation of \Next, IFIC will be responsible of the following functions:

\begin{itemize}
    \item {\bf Maintenance of the mechanical parts and gas system}:  the \Next\ gas system require scrupulous maintenance. This includes replacements and repairs of delicate parts such as the compressor, the dozen or so pumps operating at the LSC, and hundreds of gas connections (VCR fittings) that need to be periodically inspected. S. C\'arcel will be in charge of this task.
    \item {\bf Slow controls}: IFIC and UPV have jointly developed the slow controls (\bf SC) for \NEW\ and will do the same for \Next. Developing and maintenance of SC will be amongs the tasks of an electronic engineer (E1). In this project, we request funding to hire E1 during 3 years.  
    \item {\bf Data quality monitoring}: For \NEW\ IFIC was in charge of the automatic data processing system that produced data quality plots, essential to monitor the performance of the detector in real time. This responsibility will also be assumed for the operation of \Next. It will be one of the tasks of the FPI fellow requested in this project, under the supervision of and P. Novella.   
\end{itemize}

%Beyond providing shifters, IFIC will be in charge of data-quality monitoring during the detector operation (P. Novella, together with the FPI fellow requested in this project)  . The software tools devoted to these tasks, already developed by IFIC for the NEXT-White operation, will be upgraded for the data taking in \Next\ along the following lines:
%\begin{itemize}
 %   \item {\bf Data processing}: an automatised system will be set up in order to apply a first-level reconstruction of the data as soon as they are serialised into disk by the DAQ system. P. Novella will be responsible for this system.  
%    \item {\bf Data quality}: the automatic data processing will produce data quality plots to be checked by the regular shifters. This system will assure that any situation concerning the performance of the detector is identified in real time. The FPI fellow requested in this project and P. Novella will be in charge of this monitoring system.   
%\end{itemize}


\indent

\subsection*{Scientific exploitation of \Next}

As done for NEXT-White, IFIC will play a major role in the physics exploitation of \Next. In particular, the three main lines will be conducted: 

\begin{itemize}
    \item {\bf Analysis coordination}: the physics analyses will be coordinated by the Staff Scientists M. Sorel (Physics Convener of the NEXT collaboration) and P. Novella. The different tasks will be distributed among the various groups involved in NEXT. The goal of this coordination is to assure that the scientific goals (described above) are achieved within the duration of this project. 
    \item {\bf Background measurement and characterisation}: the FPI fellow requested in this project, a post-doc (PD3), M. Sorel and P. Novella will be devoted to the measurement of the radiogenic and cosmogenic backgrounds that impact the \bb\ searches in \Next. In this project, we request funding to hire PD3 during 3 years 
    \item {\bf Double beta decay analyses}: the same IFIC members will also play a leading role in the \bb\ analyses. In particular, the techniques developed in NEXT-White will be used to measure the half-life of the \bbtnu\ decay and search for the \bbonu\ process. This will be the main subject in the PhD thesis of the FPI fellow.
\end{itemize}


\subsubsection*{\HDEMO\ prototype}

\indent

IFIC will conduct the R\&D towards the design of the BFD detector and will construct the pBFD prototype. This effort, lead by J.~Mart\'in-Albo and N.~L\'opez, includes the following  tasks:
\begin{itemize}
    \item \textbf{Optical characterisation of wavelength-shifting fibres:} The main deliverable is the selection of the optimal wavelength-shifting fibres to be used in the FBD of the HD detector. Parameters to take into account include the shape of the fibers, their cladding configuration and the wavelength shifters on their surface and core. The measurements for wavelengths above 200 nm will be performed in a dark box in air using LEDs, lasers, and  a monochromator. The tests for wavelengths in the VUV region ($<200$ nm) will be done in a vacuum chamber using an ultraviolet light source (a deuterium lamp) and a vacuum monochromator (already available at IFIC).
    \item \textbf{Measurement of the photon detection efficiency of a panel of WLS fibres:} Once the optimal fibre has been selected, we will build panels with up to several dozens of them bundled together and read with a few sensors. This self-sustained panels are the configuration envisioned to be installed in the HD-DEMO detector forming a barrel. IFIC will measure the photon detection efficiency of such panels for the emission wavelength range of xenon. 
    \item \textbf{Design and construction of the pFBD:} IFIC will build the prototype fibre barrel detector (in collaboration with the groups from Israel) to be installed in HD-DEMO, measuring its performance before installation in the detector. IFIC will collaborate with DIPC and LSC in the definition of the \NHD\ interfaces and integration of \NHD, with engineering support from S.~Cárcel and a senior electronic engineer (E1).
\end{itemize}
 

\subsubsection*{Personnel in the Working Plan}
The working plan for \sIFIC\ involves 5 FTEs (S. C\'arcel, N. L\'opez-March, J. Mart\'in-Albo, P. Novella and M. Sorel) and 2 Ph.D. students (C. Romo and A. Us\'on). The responsibilities on the project of the existing permanent personnel are:

\begin{itemize}
    \item S. C\'arcel (mechanical engineer U.V.): co-coordination of the integration of \Next and responsible for its maintenance
    \item N. L\'opez-March (Ayudante Doctor U.V.): co-leader of the BFD R\&D
    \item J. Mart\'in-Albo (CSIC CIDEGENT): convener of the Monte Carlo simulation, co-leader of the BFD R\&D. 
    \item P. Novella (CSIC Staff Scientist): co-coordinator of the scientific exploitation of \Next, responsible for data-processing and quality
    \item M. Sorel (CSIC Staff Scientist): Physics convener of the NEXT collaboration, co-coordinator of the scientific exploitation of \Next.
\end{itemize}


In this project, we request for a senior electronic engineer (E1) and a post-doc (PD3). The former will carry on with the responsibilities of the IFIC group in the \Next\ slow controls, and will play a major role in the R\&D of NEXT-HD. The latter will be leading the background and \bb\ analyses.   

As the ANA subproject is specially suited for PhD students, we request for a FPI fellow.  The fellow will focus on the different aspects of the experiment, from the hardware installation to the data taking and analysis, acquiring the skills to be exploited during his/her career in particle physics. The main subject of his/her PhD thesis will be the \bb\ searches in \Next. Given the expected high-impact results, his/her role in \Next\ will
offer the fellow major visibility and acknowledgement, granting future research positions.


