\subsection{Introduction}

The NEXT program is developing the technology of high-pressure xenon gas Time Projection Chambers (TPCs) with electroluminescent amplification (\HPXeEL) for neutrinoless double beta decay searches (\bbonu)~\cite{Alvarez:2012haa, Gomez-Cadenas:2019sfa}. 

The first phase of the program included the construction, commissioning and operation of two prototypes, called NEXT-DEMO and NEXT-DBDM (with xenon masses around 1 kg), which demonstrated the robustness of the technology, its excellent energy resolution and its unique topological signature \cite{Alvarez:2012xda, Alvarez:2013gxa, Alvarez:2012hh, Ferrario:2015kta}. The \NEW\footnote{Named after Prof.~James White, our late mentor and friend.} demonstrator implemented the second phase of the program. The detector and the main results obtained through its successful operation are described in Sec.~\ref{sec.new} \Next, described in Sec.~\ref{sec.next100}, constitutes the third phase of the program. It is a radiopure detector deploying 100 kg of xenon at 15 bar and scaling up \NEW\ by slightly more than 2:1 in linear dimensions. In addition to a physics potential which is competitive with the best current experiments in the field, \Next\ can be considered as a large scale demonstrator of the suitability of the \HPXeEL\ technology for detector masses in the ton-scale. The apparatus is currently in the final phases of constructions, scheduled to start data taking in 2022.
\footnote{
Like virtually every scientific program, NEXT has felt the impact of the COVID-19 pandemic and subsequent material shortage, which has caused an accumulated delay of more than one year in NEXT-100 schedule. Fortunately, part of this impact has been mitigated by the extended operation of NEXT-White, which has allowed us to achieve to major scientific results, the first one concerning NEXT topological signature and the second concerning a novel method to measure \bb\ events
(specifically the \bbtnu\ mode) subtracting the background from the data themselves.}. 

The fourth phase of the program foresees the construction of the \NHD\ detector, deploying up to one ton of enriched xenon and described in Sec.~\ref{sec.hd}. The collaboration plan is to start construction in 2025 and operation in 2026. While the operation of the NEXT-100 will be of great value to validate most of the technical solutions to be implemented in \NHD (in the same way the \NEW\ was instrumental to \Next), substantial R\&D during this period is also required. 

The NEXT collaboration has achieved a major scientific break-through with the proposal (2016), proof of concept (2017) and recent demonstration of principle (2020) of the possibility of implementing barium tagging in a ton-scale detector. The program, called NEXT-BOLD, is described in section Sec.~\ref{sec.bold}.

{\bf In this proposal, funds are requested for the scientific exploitation of the \Next\ detector, the R\&D needed for \NHD\ and the preparation of the large infrastructures needed at the LSC for the ton-scale program.} A very intense R\&D on barium tagging will also be carried during this period. The funds for the BOLD program are secured through an ERC synergy grant (NEXT-BOLD, 2020-2026, granted to J.J. G\'omez-Cadenas, F. Coss\'io and R. Guenette). 
