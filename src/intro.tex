\subsection{Introduction}

The NEXT program is developing the technology of high-pressure xenon gas Time Projection Chambers (TPCs) with electroluminescent amplification (\HPXeEL) for neutrinoless double beta decay searches (\bbonu)~\cite{Gomez-Cadenas:2019sfa}. 

\indent

The first phase of the program included the construction, commissioning and operation of two prototypes, called NEXT-DEMO and NEXT-DBDM (with xenon masses around 1 kg), which demonstrated the robustness of the technology, its excellent energy resolution and its unique topological signature \cite{Alvarez:2012xda, Alvarez:2013gxa, Alvarez:2012hh, Ferrario:2015kta}. The \NEW\footnote{Named after Prof.~James White, our late mentor and friend.} demonstrator implemented the second phase of the program. The detector and the main results obtained through its successful operation at the LSC are described in section \ref{sec.new}. \Next, described in section \ref{sec.next100}, constitutes the third phase of the program. It is a radiopure detector deploying 100 kg of xenon at 15 bar. In addition to a physics potential which is competitive with the best current experiments in the field, \Next\ can be considered as a large scale demonstrator of the suitability of the \HPXeEL\ technology for detector masses in the ton-scale. The apparatus is currently in the final phases of construction, and scheduled to start data taking in 2022\footnote{
Like virtually every scientific program, NEXT has felt the impact of the COVID-19 pandemic and subsequent material shortage, which has caused delays in the NEXT-100 schedule. Fortunately, part of this impact has been mitigated by the extended operation of NEXT-White, which has allowed us to achieve major scientific results, the first one concerning NEXT topological signature and the second concerning a novel method to measure \bb\ events
(specifically the \bbtnu\ mode) subtracting the background from the data themselves.}. 

\indent

The fourth phase of the program foresees the construction of the \NHD\ detector\footnote{Letter of Intent submitted to the LSC Scientific Committee, September,  2021.} deploying up to one ton of enriched xenon and described in section~\ref{sec.hd}. Indeed, the operation of \Next\ will be of great value to validate many of the technical solutions to be implemented in \NHD\ (in the same way that \NEW\ was instrumental to \Next). Nevertheless, substantial R\&D is also required and proposed during this period. In particular, we propose the construction, commissioning and operation of a mid-size prototype of \NHD, the \HDEMO\ demonstrator. 

\indent

The NEXT collaboration has achieved a major scientific breakthrough with the proposal \cite{Nygren_2015}, proof of concept \cite{McDonald:2017izm} and recent demonstration of principle \cite{ rivilla_fluorescent_2020}, of the possibility of implementing barium tagging in a ton-scale detector. The program, called NEXT-BOLD, is funded through an ERC synergy grant (NEXT-BOLD, 2020-2026, granted to J.J. G\'omez-Cadenas, F. Coss\'io and R. Guenette). Although BOLD is a central ingredient in the NEXT program, it has its own funding, and will not be considered in this proposal. 
